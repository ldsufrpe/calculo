%********************************************%
%*       Generated from PreTeXt source      *%
%*       on 2020-08-25T00:36:29-03:00       *%
%*   A recent stable commit (2020-08-09):   *%
%* 98f21740783f166a773df4dc83cab5293ab63a4a *%
%*                                          *%
%*         https://pretextbook.org          *%
%*                                          *%
%********************************************%
\documentclass[oneside,10pt,]{article}
%% Custom Preamble Entries, early (use latex.preamble.early)
%% Default LaTeX packages
%%   1.  always employed (or nearly so) for some purpose, or
%%   2.  a stylewriter may assume their presence
\usepackage{geometry}
%% Some aspects of the preamble are conditional,
%% the LaTeX engine is one such determinant
\usepackage{ifthen}
%% etoolbox has a variety of modern conveniences
\usepackage{etoolbox}
\usepackage{ifxetex,ifluatex}
%% Raster graphics inclusion
\usepackage{graphicx}
%% Color support, xcolor package
%% Always loaded, for: add/delete text, author tools
%% Here, since tcolorbox loads tikz, and tikz loads xcolor
\PassOptionsToPackage{usenames,dvipsnames,svgnames,table}{xcolor}
\usepackage{xcolor}
%% begin: defined colors, via xcolor package, for styling
%% end: defined colors, via xcolor package, for styling
%% Colored boxes, and much more, though mostly styling
%% skins library provides "enhanced" skin, employing tikzpicture
%% boxes may be configured as "breakable" or "unbreakable"
%% "raster" controls grids of boxes, aka side-by-side
\usepackage{tcolorbox}
\tcbuselibrary{skins}
\tcbuselibrary{breakable}
\tcbuselibrary{raster}
%% We load some "stock" tcolorbox styles that we use a lot
%% Placement here is provisional, there will be some color work also
%% First, black on white, no border, transparent, but no assumption about titles
\tcbset{ bwminimalstyle/.style={size=minimal, boxrule=-0.3pt, frame empty,
colback=white, colbacktitle=white, coltitle=black, opacityfill=0.0} }
%% Second, bold title, run-in to text/paragraph/heading
%% Space afterwards will be controlled by environment,
%% independent of constructions of the tcb title
%% Places \blocktitlefont onto many block titles
\tcbset{ runintitlestyle/.style={fonttitle=\blocktitlefont\upshape\bfseries, attach title to upper} }
%% Spacing prior to each exercise, anywhere
\tcbset{ exercisespacingstyle/.style={before skip={1.5ex plus 0.5ex}} }
%% Spacing prior to each block
\tcbset{ blockspacingstyle/.style={before skip={2.0ex plus 0.5ex}} }
%% xparse allows the construction of more robust commands,
%% this is a necessity for isolating styling and behavior
%% The tcolorbox library of the same name loads the base library
\tcbuselibrary{xparse}
%% Hyperref should be here, but likes to be loaded late
%%
%% Inline math delimiters, \(, \), need to be robust
%% 2016-01-31:  latexrelease.sty  supersedes  fixltx2e.sty
%% If  latexrelease.sty  exists, bugfix is in kernel
%% If not, bugfix is in  fixltx2e.sty
%% See:  https://tug.org/TUGboat/tb36-3/tb114ltnews22.pdf
%% and read "Fewer fragile commands" in distribution's  latexchanges.pdf
\IfFileExists{latexrelease.sty}{}{\usepackage{fixltx2e}}
%% Footnote counters and part/chapter counters are manipulated
%% April 2018:  chngcntr  commands now integrated into the kernel,
%% but circa 2018/2019 the package would still try to redefine them,
%% so we need to do the work of loading conditionally for old kernels.
%% From version 1.1a,  chngcntr  should detect defintions made by LaTeX kernel.
\ifdefined\counterwithin
\else
    \usepackage{chngcntr}
\fi
%% Text height identically 9 inches, text width varies on point size
%% See Bringhurst 2.1.1 on measure for recommendations
%% 75 characters per line (count spaces, punctuation) is target
%% which is the upper limit of Bringhurst's recommendations
\geometry{letterpaper,total={340pt,9.0in}}
%% Custom Page Layout Adjustments (use latex.geometry)
%% This LaTeX file may be compiled with pdflatex, xelatex, or lualatex executables
%% LuaTeX is not explicitly supported, but we do accept additions from knowledgeable users
%% The conditional below provides  pdflatex  specific configuration last
%% begin: engine-specific capabilities
\ifthenelse{\boolean{xetex} \or \boolean{luatex}}{%
%% begin: xelatex and lualatex-specific default configuration
\ifxetex\usepackage{xltxtra}\fi
%% realscripts is the only part of xltxtra relevant to lualatex 
\ifluatex\usepackage{realscripts}\fi
%% end:   xelatex and lualatex-specific default configuration
}{
%% begin: pdflatex-specific default configuration
%% We assume a PreTeXt XML source file may have Unicode characters
%% and so we ask LaTeX to parse a UTF-8 encoded file
%% This may work well for accented characters in Western language,
%% but not with Greek, Asian languages, etc.
%% When this is not good enough, switch to the  xelatex  engine
%% where Unicode is better supported (encouraged, even)
\usepackage[utf8]{inputenc}
%% end: pdflatex-specific default configuration
}
%% end:   engine-specific capabilities
%%
%% Fonts.  Conditional on LaTex engine employed.
%% Default Text Font: The Latin Modern fonts are
%% "enhanced versions of the [original TeX] Computer Modern fonts."
%% We use them as the default text font for PreTeXt output.
%% Automatic Font Control
%% Portions of a document, are, or may, be affected by defined commands
%% These are perhaps more flexible when using  xelatex  rather than  pdflatex
%% The following definitions are meant to be re-defined in a style, using \renewcommand
%% They are scoped when employed (in a TeX group), and so should not be defined with an argument
\newcommand{\divisionfont}{\relax}
\newcommand{\blocktitlefont}{\relax}
\newcommand{\contentsfont}{\relax}
\newcommand{\pagefont}{\relax}
\newcommand{\tabularfont}{\relax}
\newcommand{\xreffont}{\relax}
\newcommand{\titlepagefont}{\relax}
%%
\ifthenelse{\boolean{xetex} \or \boolean{luatex}}{%
%% begin: font setup and configuration for use with xelatex
%% Generally, xelatex is necessary for non-Western fonts
%% fontspec package provides extensive control of system fonts,
%% meaning *.otf (OpenType), and apparently *.ttf (TrueType)
%% that live *outside* your TeX/MF tree, and are controlled by your *system*
%% (it is possible that a TeX distribution will place fonts in a system location)
%%
%% The fontspec package is the best vehicle for using different fonts in  xelatex
%% So we load it always, no matter what a publisher or style might want
%%
\usepackage{fontspec}
%%
%% begin: xelatex main font ("font-xelatex-main" template)
%% Latin Modern Roman is the default font for xelatex and so is loaded with a TU encoding
%% *in the format* so we can't touch it, only perhaps adjust it later
%% in one of two ways (then known by NFSS names such as "lmr")
%% (1) via NFSS with font family names such as "lmr" and "lmss"
%% (2) via fontspec with commands like \setmainfont{Latin Modern Roman}
%% The latter requires the font to be known at the system-level by its font name,
%% but will give access to OTF font features through optional arguments
%% https://tex.stackexchange.com/questions/470008/
%% where-and-how-does-fontspec-sty-specify-the-default-font-latin-modern-roman
%% http://tex.stackexchange.com/questions/115321
%% /how-to-optimize-latin-modern-font-with-xelatex
%%
%% end:   xelatex main font ("font-xelatex-main" template)
%% begin: xelatex mono font ("font-xelatex-mono" template)
%% (conditional on non-trivial uses being present in source)
%% end:   xelatex mono font ("font-xelatex-mono" template)
%% begin: xelatex font adjustments ("font-xelatex-style" template)
%% end:   xelatex font adjustments ("font-xelatex-style" template)
%%
%% Extensive support for other languages
\usepackage{polyglossia}
%% Set main/default language based on pretext/@xml:lang value
%% Enable secondary languages based on discovery of @xml:lang values
%% Enable fonts/scripts based on discovery of @xml:lang values
%% Western languages should be ably covered by Latin Modern Roman
%% end:   font setup and configuration for use with xelatex
}{%
%% begin: font setup and configuration for use with pdflatex
%% begin: pdflatex main font ("font-pdflatex-main" template)
\usepackage{lmodern}
\usepackage[T1]{fontenc}
%% end:   pdflatex main font ("font-pdflatex-main" template)
%% begin: pdflatex mono font ("font-pdflatex-mono" template)
%% (conditional on non-trivial uses being present in source)
%% end:   pdflatex mono font ("font-pdflatex-mono" template)
%% begin: pdflatex font adjustments ("font-pdflatex-style" template)
%% end:   pdflatex font adjustments ("font-pdflatex-style" template)
%% end:   font setup and configuration for use with pdflatex
}
%% Symbols, align environment, commutative diagrams, bracket-matrix
\usepackage{amsmath}
\usepackage{amscd}
\usepackage{amssymb}
%% allow page breaks within display mathematics anywhere
%% level 4 is maximally permissive
%% this is exactly the opposite of AMSmath package philosophy
%% there are per-display, and per-equation options to control this
%% split, aligned, gathered, and alignedat are not affected
\allowdisplaybreaks[4]
%% allow more columns to a matrix
%% can make this even bigger by overriding with  latex.preamble.late  processing option
\setcounter{MaxMatrixCols}{30}
%%
%%
%% Division Titles, and Page Headers/Footers
%% titlesec package, loading "titleps" package cooperatively
%% See code comments about the necessity and purpose of "explicit" option.
%% The "newparttoc" option causes a consistent entry for parts in the ToC 
%% file, but it is only effective if there is a \titleformat for \part.
%% "pagestyles" loads the  titleps  package cooperatively.
\usepackage[explicit, newparttoc, pagestyles]{titlesec}
%% The companion titletoc package for the ToC.
\usepackage{titletoc}
%% begin: customizations of page styles via the modal "titleps-style" template
%% Designed to use commands from the LaTeX "titleps" package
\pagestyle{plain}
%% end: customizations of page styles via the modal "titleps-style" template
%%
%% Create globally-available macros to be provided for style writers
%% These are redefined for each occurence of each division
\newcommand{\divisionnameptx}{\relax}%
\newcommand{\titleptx}{\relax}%
\newcommand{\subtitleptx}{\relax}%
\newcommand{\shortitleptx}{\relax}%
\newcommand{\authorsptx}{\relax}%
\newcommand{\epigraphptx}{\relax}%
%% Create environments for possible occurences of each division
%% Environment for a PTX "section" at the level of a LaTeX "section"
\NewDocumentEnvironment{sectionptx}{mmmmmm}
{%
\renewcommand{\divisionnameptx}{Seção}%
\renewcommand{\titleptx}{#1}%
\renewcommand{\subtitleptx}{#2}%
\renewcommand{\shortitleptx}{#3}%
\renewcommand{\authorsptx}{#4}%
\renewcommand{\epigraphptx}{#5}%
\section[{#3}]{#1}%
\label{#6}%
}{}%
%% Environment for a PTX "subsection" at the level of a LaTeX "subsection"
\NewDocumentEnvironment{subsectionptx}{mmmmmm}
{%
\renewcommand{\divisionnameptx}{Subseção}%
\renewcommand{\titleptx}{#1}%
\renewcommand{\subtitleptx}{#2}%
\renewcommand{\shortitleptx}{#3}%
\renewcommand{\authorsptx}{#4}%
\renewcommand{\epigraphptx}{#5}%
\subsection[{#3}]{#1}%
\label{#6}%
}{}%
%%
%% Styles for six traditional LaTeX divisions
\titleformat{\part}[display]
{\divisionfont\Huge\bfseries\centering}{\divisionnameptx\space\thepart}{30pt}{\Huge#1}
[{\Large\centering\authorsptx}]
\titleformat{\chapter}[display]
{\divisionfont\huge\bfseries}{\divisionnameptx\space\thechapter}{20pt}{\Huge#1}
[{\Large\authorsptx}]
\titleformat{name=\chapter,numberless}[display]
{\divisionfont\huge\bfseries}{}{0pt}{#1}
[{\Large\authorsptx}]
\titlespacing*{\chapter}{0pt}{50pt}{40pt}
\titleformat{\section}[hang]
{\divisionfont\Large\bfseries}{\thesection}{1ex}{#1}
[{\large\authorsptx}]
\titleformat{name=\section,numberless}[block]
{\divisionfont\Large\bfseries}{}{0pt}{#1}
[{\large\authorsptx}]
\titlespacing*{\section}{0pt}{3.5ex plus 1ex minus .2ex}{2.3ex plus .2ex}
\titleformat{\subsection}[hang]
{\divisionfont\large\bfseries}{\thesubsection}{1ex}{#1}
[{\normalsize\authorsptx}]
\titleformat{name=\subsection,numberless}[block]
{\divisionfont\large\bfseries}{}{0pt}{#1}
[{\normalsize\authorsptx}]
\titlespacing*{\subsection}{0pt}{3.25ex plus 1ex minus .2ex}{1.5ex plus .2ex}
\titleformat{\subsubsection}[hang]
{\divisionfont\normalsize\bfseries}{\thesubsubsection}{1em}{#1}
[{\small\authorsptx}]
\titleformat{name=\subsubsection,numberless}[block]
{\divisionfont\normalsize\bfseries}{}{0pt}{#1}
[{\normalsize\authorsptx}]
\titlespacing*{\subsubsection}{0pt}{3.25ex plus 1ex minus .2ex}{1.5ex plus .2ex}
\titleformat{\paragraph}[hang]
{\divisionfont\normalsize\bfseries}{\theparagraph}{1em}{#1}
[{\small\authorsptx}]
\titleformat{name=\paragraph,numberless}[block]
{\divisionfont\normalsize\bfseries}{}{0pt}{#1}
[{\normalsize\authorsptx}]
\titlespacing*{\paragraph}{0pt}{3.25ex plus 1ex minus .2ex}{1.5em}
%%
%% Styles for five traditional LaTeX divisions
\titlecontents{part}%
[0pt]{\contentsmargin{0em}\addvspace{1pc}\contentsfont\bfseries}%
{\Large\thecontentslabel\enspace}{\Large}%
{}%
[\addvspace{.5pc}]%
\titlecontents{chapter}%
[0pt]{\contentsmargin{0em}\addvspace{1pc}\contentsfont\bfseries}%
{\large\thecontentslabel\enspace}{\large}%
{\hfill\bfseries\thecontentspage}%
[\addvspace{.5pc}]%
\dottedcontents{section}[3.8em]{\contentsfont}{2.3em}{1pc}%
\dottedcontents{subsection}[6.1em]{\contentsfont}{3.2em}{1pc}%
\dottedcontents{subsubsection}[9.3em]{\contentsfont}{4.3em}{1pc}%
%%
%% Begin: Semantic Macros
%% To preserve meaning in a LaTeX file
%%
%% \mono macro for content of "c", "cd", "tag", etc elements
%% Also used automatically in other constructions
%% Simply an alias for \texttt
%% Always defined, even if there is no need, or if a specific tt font is not loaded
\newcommand{\mono}[1]{\texttt{#1}}
%%
%% Following semantic macros are only defined here if their
%% use is required only in this specific document
%%
%% Used for inline definitions of terms
\newcommand{\terminology}[1]{\textbf{#1}}
%% End: Semantic Macros
%% Division Numbering: Chapters, Sections, Subsections, etc
%% Division numbers may be turned off at some level ("depth")
%% A section *always* has depth 1, contrary to us counting from the document root
%% The latex default is 3.  If a larger number is present here, then
%% removing this command may make some cross-references ambiguous
%% The precursor variable $numbering-maxlevel is checked for consistency in the common XSL file
\setcounter{secnumdepth}{3}
%%
%% AMS "proof" environment is no longer used, but we leave previously
%% implemented \qedhere in place, should the LaTeX be recycled
\newcommand{\qedhere}{\relax}
%%
%% A faux tcolorbox whose only purpose is to provide common numbering
%% facilities for most blocks (possibly not projects, 2D displays)
%% Controlled by  numbering.theorems.level  processing parameter
\newtcolorbox[auto counter, number within=section]{block}{}
%%
%% This document is set to number PROJECT-LIKE on a separate numbering scheme
%% So, a faux tcolorbox whose only purpose is to provide this numbering
%% Controlled by  numbering.projects.level  processing parameter
\newtcolorbox[auto counter, number within=section]{project-distinct}{}
%% A faux tcolorbox whose only purpose is to provide common numbering
%% facilities for 2D displays which are subnumbered as part of a "sidebyside"
\makeatletter
\newtcolorbox[auto counter, number within=tcb@cnt@block, number freestyle={\noexpand\thetcb@cnt@block(\noexpand\alph{\tcbcounter})}]{subdisplay}{}
\makeatother
%%
%% tcolorbox, with styles, for EXAMPLE-LIKE
%%
%% example: fairly simple numbered block/structure
\tcbset{ examplestyle/.style={bwminimalstyle, runintitlestyle, blockspacingstyle, after title={\space}, after upper={\space\space\hspace*{\stretch{1}}\(\square\)}, } }
\newtcolorbox[use counter from=block]{example}[2]{title={{Exemplo~\thetcbcounter\notblank{#1}{\space\space#1}{}}}, phantomlabel={#2}, breakable, parbox=false, after={\par}, examplestyle, }
%%
%% tcolorbox, with styles, for inline exercises
%%
%% inlineexercise: fairly simple numbered block/structure
\tcbset{ inlineexercisestyle/.style={bwminimalstyle, runintitlestyle, blockspacingstyle, after title={\space}, } }
\newtcolorbox[use counter from=block]{inlineexercise}[2]{title={{Autoavaliação~\thetcbcounter\notblank{#1}{\space\space#1}{}}}, phantomlabel={#2}, breakable, parbox=false, after={\par}, inlineexercisestyle, }
%%
%% tcolorbox, with styles, for GOAL-LIKE
%%
%% objectives: early in a subdivision, introduction/list/conclusion
\tcbset{ objectivesstyle/.style={bwminimalstyle, blockspacingstyle, fonttitle=\blocktitlefont\large\bfseries, toprule=0.1ex, toptitle=0.5ex, top=2ex, bottom=0.5ex, bottomrule=0.1ex} }
\newtcolorbox{objectives}[2]{title={#1}, phantomlabel={#2}, breakable, parbox=false, objectivesstyle}
%%
%% xparse environments for introductions and conclusions of divisions
%%
%% introduction: in a structured division
\NewDocumentEnvironment{introduction}{m}
{\notblank{#1}{\noindent\textbf{#1}\space}{}}{\par\medskip}
%%
%% tcolorbox, with styles, for miscellaneous environments
%%
%% assemblage: fairly simple un-numbered block/structure
\tcbset{ assemblagestyle/.style={size=normal, colback=white, colbacktitle=white, coltitle=black, colframe=black, rounded corners, titlerule=0.0pt, center title, fonttitle=\blocktitlefont\bfseries, blockspacingstyle, } }
\newtcolorbox{assemblage}[2]{title={\notblank{#1}{#1}{}}, phantomlabel={#2}, breakable, parbox=false, assemblagestyle}
%% Localize LaTeX supplied names (possibly none)
\renewcommand*{\abstractname}{Resumo}
%% Equation Numbering
%% Controlled by  numbering.equations.level  processing parameter
%% No adjustment here implies document-wide numbering
\numberwithin{equation}{section}
%% Footnote Numbering
%% Specified by numbering.footnotes.level
\counterwithin*{footnote}{section}
%% More flexible list management, esp. for references
%% But also for specifying labels (i.e. custom order) on nested lists
\usepackage{enumitem}
%% hyperref driver does not need to be specified, it will be detected
%% Footnote marks in tcolorbox have broken linking under
%% hyperref, so it is necessary to turn off all linking
%% It *must* be given as a package option, not with \hypersetup
\usepackage[hyperfootnotes=false]{hyperref}
%% Hyperlinking active in electronic PDFs, all links solid and blue
\hypersetup{colorlinks=true,linkcolor=blue,citecolor=blue,filecolor=blue,urlcolor=blue}
\hypersetup{pdftitle={Cálculo Integral: S2}}
%% If you manually remove hyperref, leave in this next command
\providecommand\phantomsection{}
%% Graphics Preamble Entries
  \usepackage{tikz}

  \usepackage{smartdiagram}           % for a circular diagram
  %\pgfplotsset{axis x line = middle,
               axis y line = middle}
  \usetikzlibrary{backgrounds}
  \usetikzlibrary{arrows,matrix}
  \usetikzlibrary{positioning}        % for Dave R's worksheet
  \usepackage{circuitikz}             % for Virgil P's worksheet
  \usepackage{color}
%% If tikz has been loaded, replace ampersand with \amp macro
%% extpfeil package for certain extensible arrows,
%% as also provided by MathJax extension of the same name
%% NB: this package loads mtools, which loads calc, which redefines
%%     \setlength, so it can be removed if it seems to be in the 
%%     way and your math does not use:
%%     
%%     \xtwoheadrightarrow, \xtwoheadleftarrow, \xmapsto, \xlongequal, \xtofrom
%%     
%%     we have had to be extra careful with variable thickness
%%     lines in tables, and so also load this package late
\usepackage{extpfeil}
%% Custom Preamble Entries, late (use latex.preamble.late)
%% Begin: Author-provided packages
%% (From  docinfo/latex-preamble/package  elements)
%% End: Author-provided packages
%% Begin: Author-provided macros
%% (From  docinfo/macros  element)
%% Plus three from MBX for XML characters
\newcommand{ct}[1]{\color{gray}{\text{#1}}}
\newcommand{ctm}[1]{\color{gray}{#1}}

\newcommand{\doubler}[1]{2#1}
\newcommand{\dd}{\mathrm{d}}
\newcommand{\ob}[2]{\color{gray}{\overbrace{\color{black}{#1}}^{#2}}}
\newcommand{\ub}[2]{\color{gray}{\underbrace{\color{black}{#1}}_{#2}}}
\newcommand{\integral}[2]{\displaystyle\int {#1}\,\dd {#2}}
\newcommand{\integrald}[4]{\displaystyle\int_{#2}^{#3} {#1}\,\dd {#4}}
\DeclareMathOperator{\arcsec}{arc \,sec}
\DeclareMathOperator{\sin}{sen}
\DeclareMathOperator{\arcsin}{arc \,sen}
\DeclareMathOperator{\arccos}{arc \,cos}
\DeclareMathOperator{\csc}{cossec}
\DeclareMathOperator{\tan}{tg}
\DeclareMathOperator{\arctan}{arc\,tg}
\DeclareMathOperator{\cot}{cotg}
\newcommand{\lt}{<}
\newcommand{\gt}{>}
\newcommand{\amp}{&}
%% End: Author-provided macros
%% Title page information for article
\title{Cálculo Integral: S2}
\author{Leon Silva\\
Departamento de Matemática\\
Universidade Federal Rural de Pernanbuco
}
\date{August 25, 2020}
\begin{document}
%% Target for xref to top-level element is document start
\hypertarget{x:article:calc-integral-1}{}
\maketitle
\thispagestyle{empty}
\begin{abstract}
Aqui faremos um resumo das atividades da semana.%
\end{abstract}
\begin{introduction}{}%
Aqui uma introdução será necessária ``Introduction.''%
\end{introduction}%
%
%
\typeout{************************************************}
\typeout{Seção 1 Integração por substituição}
\typeout{************************************************}
%
\begin{sectionptx}{Integração por substituição}{}{Integração por substituição}{}{}{g:section:idp1}
\begin{objectives}{Objetivos: Estrutura}{g:objectives:idp2}
Aqui vamos por os objetivos%
%
\begin{enumerate}
\item{}Utilizar a regra da cadeia.%
\item{}Utilizar substituição de variáveis para determinar integrais definidas.%
\end{enumerate}
\end{objectives}
%
%
\typeout{************************************************}
\typeout{Subseção 1.1 O método de substituição}
\typeout{************************************************}
%
\begin{subsectionptx}{O método de substituição}{}{O método de substituição}{}{}{x:subsection:subsec-integracao-substituicao}
A integração por susbstituição consiste em um método para encontrar primitivas por meio de uma substituição conveniente da variável do integrando, de tal forma, a transformar o problema de integração, inicialmente complicado, em um mais simples.%
\par
Para exemplificar o método, vamos considerar resolver a integral indefinida \(\integral{2x(x^2+1)^3}{x}\). Com criativida podemos conjecturar \footnote{depois\label{g:fn:idp3}} que \(\frac{(x^2+1)^4}{4}\)  é uma primitiva de para \(2x(x^2+1)^3\)   e verificar usando a operação de derivada combinada com a \terminology{regra da cadeia}:%
\begin{align*}
\frac{\dd}{\dd x}\left[\frac{(x^2+1)^4}{4}\right] = (x^2+1)^3(2x) \amp \text{.}
\end{align*}
Dessa forma, a solução da integral é%
\begin{equation*}
\integral{2x(x^2+1)^3}{x}= \frac{(x^2+1)^4}{4} + C \text{.}
\end{equation*}
Agora, pensando em termos da regra da cadeia, seria útil tomar \(u=(x^2+1)\) e transformar o integrando em uma função do tipo potência na variável \(u\), isto é, \(u^3\). O próximo passo é verificar se com essa mudança de variável é possível encontrar \(\dd u\), em termos da variável \(x\), de tal forma que%
\begin{equation*}
2x(x^2+1)^3\dd x=u^3 \dd u\text{.}
\end{equation*}
O que em neste caso é verdadeiro pois \(\dd u/\dd x= 2x\), que na forma diferencial se escreve \(\dd u= (2x)\dd x\) Então, usando a substituição \(u=(x^2+1)\) a integral é convertida para a forma%
\begin{align*}
\integral{2x(x^2+1)^3}{x} \amp = \integral{u^3}{u} \amp \quad \color{gray}{\text{Substituição para variável}\, u.}\\
\amp = \frac{u^4}{4} +C \amp \quad \color{gray}{\text{Regra da potência.}}\\
\amp = \frac{(x^2+1)^4}{4} + C \amp  \color{gray}{\text{Solução.}}\text{.}
\end{align*}
O processo de converter a integral na variável \(x\)  para uma integral em outra variável, digamos, \(u\) é denominado de  \terminology{método de substituição \(u\)}.%
\begin{example}{Regra da potência.}{x:example:ex-potencia-01}%
\begin{enumerate}[font=\bfseries,label=(\alph*),ref=\alph*]
\item{}\(\integral{3(3x-1)^4}{x}\)%
\item{}\(\integral{(2x+1)(x^2+x)}{x}\)%
\item{}\(\integral{3x^2\sqrt{x^3-2}}{x}\)%
\item{}\(\integral{\frac{-4x}{(1-2x^2)^2}}{x}\)%
\end{enumerate}
\end{example}
\begin{inlineexercise}{}{x:exercise:exer-potencia-01}%
Determine  \(\integral{2x(x^2+1)^8}{x}\).%
\par\smallskip%
\noindent\textbf{\blocktitlefont Dica}.\hypertarget{g:hint:idp4}{}\quad{}Revise \hyperref[x:example:ex-potencia-01]{Exemplo~{\xreffont\ref{x:example:ex-potencia-01}}}.%
\par\smallskip%
\noindent\textbf{\blocktitlefont Resposta}.\hypertarget{g:answer:idp5}{}\quad{}depois%
\end{inlineexercise}
\begin{inlineexercise}{}{x:exercise:exer-potencia-02}%
Determine  \(\integral{3x\sqrt{x^3-2}}{x}\).%
\par\smallskip%
\noindent\textbf{\blocktitlefont Dica}.\hypertarget{g:hint:idp6}{}\quad{}Revise \hyperref[x:example:ex-potencia-01]{Exemplo~{\xreffont\ref{x:example:ex-potencia-01}}}.%
\par\smallskip%
\noindent\textbf{\blocktitlefont Resposta}.\hypertarget{g:answer:idp7}{}\quad{}depois%
\end{inlineexercise}
\begin{inlineexercise}{}{x:exercise:exer-potencia-03}%
Determine  \(\integral{(3x^2+6)(x^3+6x)^2}{x}\).%
\par\smallskip%
\noindent\textbf{\blocktitlefont Dica}.\hypertarget{g:hint:idp8}{}\quad{}Revise \hyperref[x:example:ex-potencia-01]{Exemplo~{\xreffont\ref{x:example:ex-potencia-01}}}.%
\par\smallskip%
\noindent\textbf{\blocktitlefont Resposta}.\hypertarget{g:answer:idp9}{}\quad{}depois%
\end{inlineexercise}
\begin{example}{Reescrevendo o integrando.}{x:example:ex-reescrevendo-02}%
\begin{enumerate}[font=\bfseries,label=(\alph*),ref=\alph*]
\item{}Determine \(\integral{x(3-4x^2)^2}{x}\).%
\item{}criar outro exemplo...%
\end{enumerate}
\end{example}
\begin{example}{Nem sempre é possível.}{x:example:ex-subs-03}%
Determine \(\integral{-8(3-4x^2)^2}{x} .\)%
\par\smallskip%
\noindent\textbf{\blocktitlefont Solução}.\hypertarget{g:solution:idp10}{}\quad{}A escolha \(u=3-4x^2\)  gera \(\dd u/\dd x=-8x \), isto é, \(du=(-8x)\dd x\). Note a escolha de \(u\) quando combinada com \(\dd u\), não permite o usar o método de substituição, uma vez que não produz uma integral em relação a \(u\). No  \hyperref[x:example:ex-reescrevendo-02]{Exemplo~{\xreffont\ref{x:example:ex-reescrevendo-02}}} rescrevemos o integrando usando a estratégia de multiplicar e dividir por uma constante e então retirar a constante do integrando. O fato é que isso não é possível quando trata-se de variáveis já que%
\begin{equation*}
\integral{-8(3-4x^2)^2}{x} \neq \frac{1}{x}\integral{(3-4x^2)^2(-8x)}{x}.
\end{equation*}
Nesse caso a melhor estégia é expandir o integrando para gerar um polinômio e depois aplicar a regra da potência em cada parcela.%
\begin{align*}
\integral{-8(3-4x^2)^2}{x}  \amp = \integral{(-128x^4+192x^2-72)}{dx} \\
\amp = -72x+64x^3-\frac{128}{5}x^5 + C \text{.}
\end{align*}
%
\end{example}
\begin{example}{Envolvendo funções trigonométricas.}{x:example:ex-subs-04}%
\begin{enumerate}[font=\bfseries,label=(\alph*),ref=\alph*]
\item{}Determine \(\integral{\sin{(x+9)}}{x}\)%
\item{}Determine \(\integral{\cos{(5x)}}{x}\).%
\end{enumerate}
\end{example}
\begin{example}{Envolvendo funções exponenciais.}{x:example:ex-subs-05}%
\begin{enumerate}[font=\bfseries,label=(\alph*),ref=\alph*]
\item{}Determine \(\integral{\frac{e^{\sqrt{x}}}{\sqrt{x}}}{x}\).%
\item{}Determine \(\integral{\frac{e^x}{\sqrt{1-e^{2x}}}}{x}\).%
\end{enumerate}
\end{example}
\begin{assemblage}{Roteiro do método de substituição.}{x:assemblage:assemblage-roteiro}%
%
\begin{itemize}[label=\textbullet]
\item{}passo 1%
\item{}passo 2%
\item{}Passo 3%
\end{itemize}
%
\end{assemblage}
\begin{inlineexercise}{}{x:exercise:exer-subsu-01}%
depois%
\par\smallskip%
\noindent\textbf{\blocktitlefont Dica}.\hypertarget{g:hint:idp11}{}\quad{}depois%
\par\smallskip%
\noindent\textbf{\blocktitlefont Resposta}.\hypertarget{g:answer:idp12}{}\quad{}depois%
\end{inlineexercise}
\begin{inlineexercise}{}{x:exercise:exer-subsu-02}%
depois%
\par\smallskip%
\noindent\textbf{\blocktitlefont Dica}.\hypertarget{g:hint:idp13}{}\quad{}depois%
\par\smallskip%
\noindent\textbf{\blocktitlefont Resposta}.\hypertarget{g:answer:idp14}{}\quad{}depois%
\end{inlineexercise}
\begin{example}{Integração por substituição.}{x:example:ex-subsu-05}%
\begin{enumerate}[font=\bfseries,label=(\alph*),ref=\alph*]
\item{}teste%
\item{}teste%
\end{enumerate}
\end{example}
\begin{inlineexercise}{}{x:exercise:exer-regras-01}%
depois%
\par\smallskip%
\noindent\textbf{\blocktitlefont Dica}.\hypertarget{g:hint:idp15}{}\quad{}depois%
\par\smallskip%
\noindent\textbf{\blocktitlefont Resposta}.\hypertarget{g:answer:idp16}{}\quad{}depois%
\end{inlineexercise}
\end{subsectionptx}
%
%
\typeout{************************************************}
\typeout{Subseção 1.2 Substituições Trigonométricas}
\typeout{************************************************}
%
\begin{subsectionptx}{Substituições Trigonométricas}{}{Substituições Trigonométricas}{}{}{x:subsection:subsec-substituicoes-trigonometricas}
\begin{example}{Usando seno.}{x:example:ex-seno-01}%
\begin{enumerate}[font=\bfseries,label=(\alph*),ref=\alph*]
\item{}teste%
\item{}teste%
\end{enumerate}
\end{example}
\begin{example}{Usando Tangente.}{x:example:ex-tangente-02}%
\begin{enumerate}[font=\bfseries,label=(\alph*),ref=\alph*]
\item{}teste%
\item{}teste%
\end{enumerate}
\end{example}
\begin{example}{Completando quadrados.}{x:example:ex-outras-03}%
\begin{enumerate}[font=\bfseries,label=(\alph*),ref=\alph*]
\item{}teste%
\item{}teste%
\end{enumerate}
\end{example}
\begin{inlineexercise}{}{x:exercise:exer-substrig-01}%
depois%
\par\smallskip%
\noindent\textbf{\blocktitlefont Dica}.\hypertarget{g:hint:idp17}{}\quad{}depois%
\par\smallskip%
\noindent\textbf{\blocktitlefont Resposta}.\hypertarget{g:answer:idp18}{}\quad{}depois%
\end{inlineexercise}
\begin{inlineexercise}{}{x:exercise:exer-substrig-02}%
depois%
\par\smallskip%
\noindent\textbf{\blocktitlefont Dica}.\hypertarget{g:hint:idp19}{}\quad{}depois%
\par\smallskip%
\noindent\textbf{\blocktitlefont Resposta}.\hypertarget{g:answer:idp20}{}\quad{}depois%
\end{inlineexercise}
\end{subsectionptx}
\end{sectionptx}
\end{document}